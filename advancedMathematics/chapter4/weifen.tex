\documentclass[a4paper,11pt]{book}
\usepackage{xeCJK}
\usepackage{amsmath}
\usepackage[hmargin=1.25in,vmargin=1in]{geometry}
\usepackage{tikz}
\usepackage{color}
\usepackage{graphicx}
\begin{document}



\chapter*{第四章 知识框架 \quad 线性微分方程}  

\[
\text{一阶}    
\begin{cases}
    \text{可分离变量(首选)}\\
    \text{看形式}
        \begin{cases}
            \text{齐次方程(分子分母每个元素都同次)令}\frac{y}{x}=u,\mathrm{d}y=\mathrm{d}ux=x\mathrm{d}u+u\mathrm{d}x\\
            \text{公式法}y^{\prime}+P(x)y=Q(x)\\
            \text{其他换元,如令}x+y=u,y=\sin t
        \end{cases}\\
    \text{颠倒自变量和因变量}\\
\end{cases}
\]


\[
\text{二阶}
    \begin{cases}
        \text{可降阶的:}y^{\prime \prime}=f(x)\\
        \text{颠倒自变量和因变量(用到反函数求二阶导的公式)}\\
        \text{特征方程求对应的齐次方程的通解+ \underline{待定系数法}求特解(只适用于常系数)}\\
        \qquad \quad \qquad \qquad \qquad \qquad \qquad \qquad \text{也可用算子解法}\\
        \qquad \quad \qquad \qquad  \text{求出来的特解在每一个其次方程的解里面都存在且系数为1}\\
    \end{cases}    
\]

\vspace{4ex}

1. \quad $y^{\prime\prime}+py^{\prime}+qy=e^{\lambda x}P_{m}(x)$

\quad \quad 特解设为$y^{*}=x^{k}R_{m}^{(x)}e^{\lambda x}$

\vspace{2ex}

\quad \quad m代表m次的项式

\[
k
\begin{cases}
    0 \quad \lambda\text{不是}r^{2}+pr+q=0\text{的解}\\
    1 \quad \lambda\text{是}r^{2}+pr+q=0\text{的其中一个解}\\
    2 \quad \lambda\text{是}r^{2}+pr+q=0\text{的重根}
\end{cases}    
\]    


\begin{tabbing}
    1.若$P_{m}(x)=2x+1$ \hspace{6ex} \= $R_{m}(x)$设为$ax+b$\\
    2.若$P_{m}(x)=5x^{2}+6x+7$ \> $R_{m}(x)$设为$a^{2}+b^{2}+c$\\
    3.若$P_{m}(x)=3$ \> $R_{m}(x)$设为a
\end{tabbing}

然后把$y^{*}$当作y,代入$\left(y^{*}\right)^{\prime \prime}+p\left(y^{*}\right)^{\prime}+q y^{*}=e^{\lambda x}P_{m}(x)$求出这些a,b,c

2.







\end{document}