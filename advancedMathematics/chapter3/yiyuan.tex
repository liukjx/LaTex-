\documentclass[a4paper,11pt]{book}
\usepackage{xeCJK}
\usepackage{amsmath}
\usepackage[hmargin=1.25in,vmargin=1in]{geometry}
\begin{document}

\chapter*{第三章 知识框架 \quad 一元积分学}
\begin{center}
    数一为常规考点,数二,数三为考查重点    
\end{center}

\[
\text{\ }
\begin{cases}
    \text{不定积分}
        \begin{cases}
            \text{两类换元法}
                \begin{cases}
                    \text{第一类(不定积分考的多)}\\
                    \text{第二类(定积分,定积分不定积分都有,比较难的是三角分式有理分式)}

                \end{cases}\\
            \text{分部积分(多在证明题中出现)}\\
            \text{原函数的含义}
        \end{cases}\\
    \text{定积分}
        \begin{cases}
            \text{可积的定义与图像}\\
            \text{计算}
                \begin{cases}
                    \text{常规型}\\
                    \text{带有极限符号的}
                \end{cases}\\
            \text{证明题(最常见的一般是构造函数,因为变上线积分也是一种函数)}
        \end{cases}\\
    \text{反常积分}
        \begin{cases}
            \text{反常积分的计算}\\
            \text{判断敛散性,选择题考(数三不考)}
        \end{cases}\\
    \text{定积分的应用}
        \begin{cases}
            \text{几何应用(考试热点,常在填空题或大题中与微分方程一起考察)}\\
            \text{物理应用(数三不考)}\\
            \text{经济应用(数一,二不考)}
        \end{cases}\\
\end{cases}    
\]

\section*{一 \quad 误区}

\quad \ 很多人认为在某一区间上存在原函数和可积是同一回事,
其实是不对的

原函数是相对不定积分而言的

可积是相对于定积分而言,即定积分的值存在

\noindent 那么

1.什么情况下不存在原函数呢?

有一类间断点时一定不存在原函数,
    有第二类间断点时,
有可能存在原函数,也有可能没有.


比如$\displaystyle{f(x)=\left\{\begin{array}{ll}
    2 x \sin \frac{1}{x}-\cos \frac{1}{x} & (x \neq 0) \\
    0 & x=0
    \end{array}\right.}$有原函数,为
$\displaystyle{F(x)=\left\{\begin{array}{l}
        x^{2} \sin \frac{1}{x}  \quad x \neq 0 \\
        0 \qquad \qquad x=0
        \end{array}\right.}$

2.什么情况下不可积?($\int _{a}^{b}f(x)\mathrm{d}x$的值存在)
\begin{enumerate}
    \item f(x)在$[a,b]$上连续,则可积
    
    \item f(x)在$[a,b]$上有界,且只有有限个间断点(可去,跳跃,震荡有界均可)则可积
    
    \item 有无穷间断点的时候,有可能可积也有可能不可积(这就是要变了了瑕积分问题)
    
\end{enumerate}



\section*{二 \quad 不定积分}

\[
\text{换元法}    
\begin{cases}
    \text{第一类:临场发挥,随机拼凑}\\
    \text{第二类}
        \begin{cases}
            \text{三角换元(其中n为常数)}
                \begin{cases}
                    \text{出现}\left(a^{2}-x^{2}\right)^{m}\text{型,经常令}x=a\sin t\\
                    \text{出现}\left(a^{2}+x^{2}\right)^{m}\text{型,经常令}x=a\tan t\\
                    \text{出现}\left(x^{2}-a^{2}\right)^{m}\text{型,经常令}x=a \sec t\\
                    \text{出现形如}(\arcsin x)^{m},(\arctan x)^{m}\text{的反函数时,}\\
                    \text{经常令}x=\sin t \text{或}x=\tan t\\
                \end{cases}\\  
            \text{倒代换:出现头重脚轻(分母次数比分子高很多时)}\text{可以考虑使用}x=\frac{1}{t}
        \end{cases}\\
\end{cases}
\]

分部积分:“指三幂对反”通常情况下谁在前面谁充当$v^{\prime}$(特别地,有些分部积分求不出来,但可以内部抵掉,或是
产生一个负的相同项后转到等号左侧)

注:第一类换元法:

1.当出现$\int R (\sqrt[n]{ax+b},\sqrt[m]{ax+b})\mathrm{d}x$型时,经常令

\[
  \sqrt[k]{ax+b} =t,k\text{为}n,m\text{的最小公倍数} 
\]

2.当出现$\displaystyle{\int R\left(x,\sqrt{\frac{ax+b}{cx+d}}\right)\mathrm{d}x}$型,经常令$\displaystyle{\sqrt{\frac{ax+b}{cx+d}}=t}$

3.对于$\sin ^{2k}x\cos ^{2l}x,(k,l \in N)$型函数,总可利用三角恒等式化成cos2x的多项式

\[
    \sin ^{2} x=\frac{1}{2}(1-\cos 2 x)
\]

\[
    \cos ^{2} x=\frac{1}{2}(1+\cos 2 x)  
\]

4.对于$\tan ^{n} x \sec ^{2 k} x$或$\tan ^{2 k-1} x \sec ^{n} x\left(n, k, \in N_{+}\right)$型函数的积分可依次作变换
$u=\tan x$或$u=\sec x$求得结果

\section*{专题一 \quad 分式的不定积分问题}
\[
\text{分式}
\begin{cases}
    \text{有理分式}\\
    \text{三角分式}
\end{cases}    
\]

\subsection*{一:有理分式}



\emph{步骤一}:观察,如果很容易拆项,裂项,直接拆开变成简单或可以使用全书上总结的可以带公式的分式

\vspace{2ex}

\noindent  \emph{步骤二}:观察不出来时,先判断是否是真假分式,如果是假分式,先将其拆成一个真分式加一个多项式

\vspace{2ex}

\noindent  \emph{步骤三}:处理真分式
    \begin{enumerate}
        \item 若分母含有$(x-a)^{m}$,分解为$\displaystyle{\frac{A_{1}}{x-a}-\frac{A_{2}}{(x-a)^{2}}+\cdots +\frac{A_{m}}{(x-a)m}}$
        \item 若分母中含有$(ax^{2}+bx+c)^{m}$且$b^{2}-4ac<0$,则分解为$\displaystyle{\frac{A_{1} x+B_{1}}{\theta x^{2}+b x+c}+\frac{A_{2} x+B_{2}}{\left(a x^{2}+b x+c\right)^{2}}}+\cdots+\frac{A _{m} x+B m}{\left(a x^{2}+b x-c\right)^{m}}(m \text{很少}\ge 2)$
    \end{enumerate}

\vspace{2ex}

注:步骤一二可能共存,关键在于求出$A_{1},A_{2},\cdots,A_{n},B_{1},B_{2},\cdots,B_{n}$这些分数,可以用待定系数法求出这些系数,也可以用拉普拉斯方程中的方法

(如果步骤三失效,可尝试用倒代换等其他方法)

\subsection*{二:三角分式}

\begin{enumerate}
    \item 半角化
          \[
          \text{半角公式}
          \begin{cases}
              \sin 2x = 2 \sin x \cos x\\
              \cos 2x = 2 \cos ^{2} x -1 =1-2\sin ^{2} x=\cos ^{2}x-\sin ^{2}x
          \end{cases}    
          \]
    \item $\displaystyle{\frac{c\sin x+d \cos x}{a \sin x +b\cos x}}$型
    \item 万能公式(转化成幂函数分式)
    \item 若出现$a \cos x +b \sin x$可以变成$\sqrt{a^{2}+b^{2}}\sin (x + \theta),\theta = \arctan \frac{a}{b}$
    \item 积化和差,可能会用到,但和差化积比较少,因为乘法更复杂
    \item 把被积函数变成$f(\tan x)$ \quad $\mathrm{d}x$变成$\mathrm{d}\tan x$,
    
    常用到半角公式$1+\sin 2x =(\sin x +\cos)^{2},1=\sin^{2}+\cos ^{2}x$
\end{enumerate}

\section*{定积分补充公式}

\begin{enumerate}
    \item $\displaystyle{\left|\int_{a}^{b} f(x) d x\right| \le \int_{a}^{b} | f(x) | d x \quad(a \leq b)}$
    \item 柯西不等式 $\displaystyle{\left(\int_{a}^{b} f(x) g(x) d x\right)^{2} \le \int_{a}^{b} f^{2}(x) d x \int_{a}^{b} g^{2}(x) d x}$
    \item 定积分中值定理 $\displaystyle{\int_{a}^{b} f(x) d x=f(\xi)(b-a)}$用介值定理证明$\xi \in [a,b]$
    设$\displaystyle{F(x)=\int_{a}^{x} f(t) d t}$当$F(x)$在[a,b]上满足拉格朗日定理的条件时,即F(x)在[a,b]上连续,在(a,b)上可导
    ,则$\exists \xi \in (a,b)$使得$F(b)-F(a)=F^{\prime}(\xi)(b-a)$其中$F(b)=\int _{a}^{b} (x)\mathrm{d}x\quad F(a)=0 \quad F^{\prime}(\xi)=f(\xi)$
    %上面这句后面的公式不是太明白什么意思

    拓展形式:$\int_{a}^{b} f(x) g(x) d x=f(\xi) \int_{a}^{b} g(x) d x(g(x)\text{不变号},\xi \in [a,b])$
    \item $\displaystyle{\int_{0}^{\frac{\pi}{2}} f(\sin x) d x=\int_{0}^{\frac{\pi}{2}} f(\cos x) d x}$
    
    引申华里士公式$\displaystyle{\int_{0}^{\frac{\pi}{2}} \sin ^{n} x \mathrm{d} x=\int_{0}^{\frac{\pi}{2}} \cos ^{n} x \mathrm{d} x=\left\{\begin{array}{ll}
        \frac{n-1}{n} \cdot \frac{n-3}{n-2} \cdot \frac{n-5}{n-4} \cdot \cdots \cdot \frac{2}{3}, & n \ge 3 \text { 为奇数 } \\
        \frac{n-1}{n} \cdot \frac{n-3}{n-2} \cdot \frac{n-5}{n-4} \cdot \cdots \cdot \frac{1}{2} \cdot \frac{\pi}{2}, & n \ge 2 \text { 为偶数 }
        \end{array}\right.}$
    \item $\displaystyle{\int_{0}^{\pi} f(\sin x) d x=\int_{0}^{\frac{\pi}{2}} f(\cos x) d x}$
    \item $\displaystyle{\int_{0}^{\pi} x f(\sin x) d x}=\frac{\pi}{2} \int_{0}^{\pi} f(\sin x) d x$
    
    引申:(1)$\displaystyle{\int_{0}^{\pi} x f(|\cos x|) d x=\frac{\pi}{2} \int_{0}^{\pi} f(|\cos x|) d x}$

    \qquad \quad (2)$\displaystyle{\int_{0}^{\pi} x f\left(\cos ^{2} x\right) d x=\frac{\pi}{2} \int_{0}^{\pi} f\left(\cos ^{2} x\right) d x}$


    \item 同期函数的定积分性质
    
    \vspace{2ex}
    
    \quad \qquad $\displaystyle{\int_{n}^{a+T} f(x) d x=\int_{0}^{T} f(x) d x}$

    \quad \qquad $\displaystyle{\int_{0}^{n \pi} f(x) d x=n \int_{0}^{T} f(x) d x}$

    \quad \qquad 引申1 $\displaystyle{\int_{a}^{a+n T} f(x) d x=n \int_{0}^{T} f(x) d x}$

\end{enumerate}

\section*{定积分的计算题}

\[
    \begin{array}{c}
        \text{带有具体被积函数表达式的定积分计算}\\
        \text{先看有无瑕点和无穷区间}\\
        \text{有则归为反常积分的计算}
    \end{array}
    \begin{cases}
        \text{1.首先看有无奇偶可用,简化运算}\\
        \text{2.其次看能否用讲过的三角函数,周期函数公式来简化运算}\\
        \text{3.对于第一类换元法,常用的换元法是}t=a+b-x\\
        \quad \text{特别是对称区间时,经常对其中半个区间使用,就可以聚合}\\
        \quad \text{成一个区间}(t=a+b-x)\text{换元法经常用在证明积分大(小)于}\\
        \quad \text{0的证明题中间可能不对称}\\
        %中间可能不对称是什么意思

        \text{4.第一类换元中的其他方法,第二类换元法,分部积分}\\
        \quad \text{幂函数分式,三角分式和计算不定积分一样,先求出原}\\
        \quad \text{函数,然后带入上下限,注意上下限有时候要相应的改变}\\

        \text{5.被积函数中带有绝对值}\min {},\max {}\text{要注意重新划分区间}\\
        \quad \text{因为不同区间内被积函数表达式不同}\\

        \text{6.转化成二重积分再做计算(有时候要改变积分次序)}\\
    \end{cases}
\]

注:1.2.3.6是定积分才有,不定积分没有的

带有定积分的求极限:问题(不要使用积分中值定理),极限有可能加在定积分号的外面,也有可能加在被积函数上:直接使用夹逼准则也就是放缩被积函数

\section*{定积分的证明题(1.2.3.4.5.6都可能用在定积分不等式中)}

\[
\text{\ }    
\begin{cases}
    \text{1.比较带有具体函数表达式的定积分大小问题}\\
    \text{2.构造函数,配合求导,求最值等方法证明定积分不等式}\\
    \quad \text{特别地如果含a,b两个常数,经常把其中一个设为x,以前积分里的x变成t在讨论}
    \text{3.使用放缩证明定积分不等式}
        \begin{cases}
            \text{放缩被积函数}\\
            \text{放缩被积分区间(一般是缩小)}\\
        \end{cases}\\
    \text{4.对被积函数进行拉格朗日或泰勒展开}\\
    \text{5.使用公式证明}
        \begin{cases}
            \text{不等式}
                \begin{cases}
                    \text{柯西施瓦茨不等式}\\
                    \text{绝对值不等式}\left|\int_{a}^{b} f(x) d x\right| \le \int_{a}^{b}|f(x)| d x\\
                    \text{当} f(x)\ge g(x) \text{时}\int_{a}^{b} f(x) d x \geq \int_{a}^{b} g(x) d x,(b>a)
                \end{cases}\\
            \text{公式:补充的周期函数,三角函数,积分中值定理及拓展形式}
        \end{cases}\\
    \text{6.转化成二重积分再证明}\\
    \text{7.定积分的零点类问题}
        \begin{cases}
            \text{法一:写出定积分的变上限函数形式,变上限函数也是函数}
                \begin{cases}
                    \text{零点定理}\\
                    \text{罗尔定理}
                \end{cases}\\
            \text{法二:使用积分中值定理(一般是证明题中的辅助手段)}\\
        \end{cases}\\
\end{cases}
\]


注:变上限函数的零点问题又回归到之前讲的微分学中的零点问题了,零点和罗尔都只能证明至少有一个零点,如果还要证明零点是唯一的,则需要通过求导说明其是单调的,罗尔定理经常使用我们前面讲过的还原法

\vspace{2ex}

2中含a,b两个参数的证明不等式问题,经常令其中一个常数设成x,另一个不动,然后借助求导求最值的方法,很类似之前的不等式专题中的法二不能分离参数的形式

\section*{小专题一}

已知f(x)和$\int_{a}^{b} f(x) g(x) d x$的关系(g(x)是具体函数,有时候取1),求f(x)的函数表达式

步骤一:设$\int_{a}^{b} f(x) g(x) d x=A$

步骤二:对整个函数等式左右两侧同乘以g(x)后,再取从a到b的积分,然后通过该方程解出未知数A,从而得f(x)的表达式

\vspace{2ex}

\noindent 拓展1:已知f(x)同时和$\int _{a}^{b}f(x)d x \quad \int _{c}^{d}f(x)dx$的函数关系,让求f(x)?

步骤一:设$\int _{a}^{b}f(x)dx=A \quad \int _{c}^{d}f(x)dx=B$

步骤二:先对整个函数等式取从a到b的积分,然后再对整个函数取从c到d的积分,得到一个一元二次方程,解出来未知数A,B,从而得到f(x)的表达式

\vspace{2ex}

\noindent 拓展2:有些重积分也会用到该专题

\section*{小专题二}

比较有具体被积函数表达式的定积分大小

\[
\text{\ }    
\begin{cases}
    \text{定积分}
        \begin{cases}
            \text{比较被积函数大小(前提上限大于下限)}\\
            \text{做差合并区间}
        \end{cases}\\
    \text{重积分(选择题居多)}
        \begin{cases}
            \text{积分区域是一样时,比较被积函数的大小}\\
            \text{被积函数相同时,比较积分区域的大小}
        \end{cases}
\end{cases}
\]

\section*{定积分比大小}

法一:如果可以先借助奇偶性简化,然后根据$f(x)>g(x)$则
$\displaystyle{\int _{a}^{b}f(x)dx > \int _{a}^{b}g(x)dx}$这个定理来判断,实际上就是比较被积函数的大小

\vspace{2ex}

\noindent 法一:对积分区间分段,通过换元把不同区间变成同一个区间,然后合并积分,在新区间内的被积函数恒大于0

\section*{小专题三}

利用分部积分证明或求定积分

情形一:题干已知$f(x),f^{\prime}(x)\cdots$的某些具体数值,被积函数中含有更高阶的导数,采用分部积分$\int u v^{\prime}dx=uv-\int u^{\prime}vdx$连续让那个更高阶的导数充当$v^{\prime}$,每次得到$uv$后代入题干中的某些条件,可能要用多次分部积分

例一:f(x)在[a,b]上二阶连续时,又$f(a)=f^{\prime}(a)=0$

求证:$\displaystyle{\frac{1}{2} \int _{a}^{b}f^{\prime \prime}(x)(x-b)^{2}dx=\int _{a}^{b}f(x)dx}$

例二:$f(0)=f(1)=0\quad$ 求证$\displaystyle{\int _{0}^{1}f(x)dx=-\frac{1}{2}\int _{0}^{1}(x-x^{2})f^{\prime\prime}(x)dx}$

情形二:题干有变上限函数,让球的定积分被积函数中含有变上限函数

采用分部积分$\int u v^{\prime}dx=uv-\int u^{\prime} v dx$让那么变上限函数充当u,其余部分充当$v^{\prime}$

例一:设$\displaystyle{G(x)=\int _{1}^{x^{2}}\frac{t}{\sqrt{1+t^{3}}}dt}$,则$\displaystyle{\int _{0}^{1}xG(x)dx=}$\underline{\hbox to 10mm{}}

例二:设$G^{\quad \prime}=\arcsin (x-1)^{2},G(0)=0$求$\displaystyle{\int _{0}^{1}G(x)dx=}$\underline{\hbox to 10mm{}}

\section*{小专题四 \quad 反常积分的计算}

步骤一:首先看能不能对$\int _{0}^{+\infty}$ 上的函数使用$\Gamma(x)$函数的公式,或着变形成$\Gamma(x)$后直接套用关于$\Gamma(x)$的现成公式

\vspace{2ex}

\noindent 步骤二:若不能套公式看区间内部有没有瑕点

\noindent 步骤三:

1.如果内部无瑕点,直接装不知道,按定积分计算,求出原函数,带入端点值相减(如果端点处是瑕点或无穷,则端点处的反常积分实则就是求极限问题,如果两个极限有一个或两个不存在,就是发散的)

2.如果内部有瑕点,必须以瑕点为中断点,把区间沿瑕点,断成两个,逐个区间装不知道,使用求定积分的办法

\section*{小专题五 \quad 审敛法修改版}

1.设$f(x)$在区间$[a,+\infty]([-\infty,b])$上连续,如果$\exists$P>1

\vspace{2ex}

使得$\displaystyle{\lim \limits_{\substack{x \rightarrow+\infty\\x \rightarrow -\infty}} x^{p} f(x)=}$常数 , 那么收敛

\vspace{2ex}

如果$\displaystyle{\lim \limits_{\substack{x \rightarrow+\infty\\x \rightarrow -\infty}} x f(x)=}$ 等于不为0的常数或为无穷,那么反常积分发散

注:若$x^{p}f(x)\rightarrow \infty$或$xf(x)=0$极限审敛法失效,敛散性判别需改换方式重判

\vspace{2ex}

\noindent 2.$f(x)$在[a,b]上连续,如果$\exists q$使$\lim \limits_{x \rightarrow a^{+}}(x-a)^{q}f(x)$(或$\lim \limits_{x \rightarrow b^{-}}(b-x)^{q}f(x))$存在,那就收敛

\vspace{2ex}

如果$\lim \limits _{x \rightarrow a^{+}}(x-a)f(x)$(或$\lim \limits _{x \rightarrow b^{-}}(b-x)f(x)$)等于不为0的常数或趋于无穷,那就发散

(数三不要求掌握)

\section*{小专题六 \quad 定积分的应用}

\vspace{3ex}

\[
\text{\ }    
\begin{cases}
    \text{几何}
        \begin{cases}
            \text{面积}
                \begin{cases}
                    \text{直角坐标}\\
                    \text{极坐标}
                \end{cases}\\
            \text{弧长}
                \begin{cases}
                    \text{一般参数式}\\
                    \text{直角坐标}\\
                    \text{极坐标}
                \end{cases}\\
            \text{曲面积}\\
            \text{体积}
                \begin{cases}
                    \text{旋转体体积}
                        \begin{cases}
                            \text{模型一:圆纸片的叠加(圆台模型)}\\
                            \text{模型二:卷状卫生纸模型}
                        \end{cases}\\
                    \text{平行截面积已知的立体体积}\\
                \end{cases}\\
            \text{平均值}\frac{\int _{a}^{b}f(x)dx}{b-a}\\
        \end{cases}\\
    \text{物理}
        \begin{cases}
            \text{质量}\\
            \text{形心,质心(二重积分有公式)}\\
            \text{力}
                \begin{cases}
                    \text{引力}a\frac{m_{1}m_{2}}{r^{2}}\\
                    \text{压力}PS\\
                    \text{弹力}F=kx\\
                    \text{重力}mg
                \end{cases}\\
            \text{功}W=F·S=FS\cos \theta
        \end{cases}\\
    \text{经济(数一,二不考)}\\
\end{cases}
\]

\section*{1.平面中的三种坐标形式}

1.直角坐标:y=y(x)
2.参数式$\begin{array}{l}
    x=x(t) \\
    y=y(t)
    \end{array}$
3.极坐标$\rho=\rho(\theta)$

例如$x^{2}+y^{2}=1$通过$\begin{array}{l}
    x=\rho \cos \theta \\
    y=\rho \sin \theta
    \end{array}$
的极坐标转化成$\rho = 1 $的极坐标形式,其中有两个$\rho \theta$未知数

\section*{几种常见图形}

1.圆$x^{2}+y^{2}=a^{2}$,参数方程形式$\begin{array}{c}
x=a\cos \theta \\
y=b \sin \theta 
\end{array}
$$(0 \le \theta \le 2\pi)$

\noindent 2.椭圆:$\displaystyle{}$参数方程形式$\begin{array}{c}
    x=a\cos \theta \\
    y=b \sin \theta 
    \end{array}
    $$(0 \le \theta \le 2\pi)$


\noindent 3.摆线:参数方程$\displaystyle{\begin{array}{l}
    x=a(t-\sin t) \\
    y=a(1-\cos t)
    \end{array}}$

\noindent 4.星形线 $x^{\frac{2}{3}}+y^{\frac{2}{3}}=a^{\frac{2}{3}}$参数的方程$\begin{array}{l}
    x=a \cos ^{3} \theta \\
    y=a \sin ^{3} \theta
    \end{array}$$(\cos \theta \le 2 \pi)$
    
\noindent 5. 心形线 $a(1+\cos \theta) (a>0)$上半部分为$[0,\pi]$
$S=2S_{A1}=2\int _{0}^{\pi} \frac{1}{2}\theta^{2}(1+\cos \theta)^{2}d \theta \quad l=2l_{1}$

\noindent 6.伯努利双扭线(双重积分中考察,直接转化成极坐标计算)

$x^{2}+y^{2}=a^{2}(x^{2}-y^{2})$极坐标形式$\begin{array}{l}
    x=\rho \cos \theta \\
    y=\rho \sin \theta
    \end{array}\Rightarrow \rho ^{2}=a^{2}\cos 2\theta$


    $\theta \in \left(0, \frac{\pi}{4}\right) \cup\left(\frac{3}{4} \pi, \frac{5}{4} \pi\right) \cup\left(\frac{7}{4} \pi, 2 \pi\right)$

综上5.6有极坐标形式,但没有参数方程

\section*{旋转体体积}
\subsection*{圆台体模型}

由连续曲线$y=f(x)$与$x=a,x=b$及x轴所围成的曲边梯形,绕x轴旋转一周所得的旋转体体积

\vspace{2ex}

$\displaystyle{V_{x}=\int_{a}^{b} \pi f^{2}(x) d x \quad(a<b)}$

\vspace{2ex}

引申1:若绕y=a旋转一周呢?

\vspace{2ex}

$\displaystyle{V_{x}=\int_{a}^{b} \pi[f(x)-a]^{2} d x}$

\vspace{2ex}

引申2:如果同时存在$f_{2}(x), f_{1}(x)$且$f_{2}(x)\ge f_{1}(x) \ge 0$呢?
   
\vspace{2ex}

$\displaystyle{V=V_{2}-V_{1}=\pi \int_{a}^{b} f_{2}^{2}(x) d x-\pi \int_{a}^{b} f_{1}^{2}(x) d x}$

\vspace{2ex}

二,如果绕y轴旋转一周,所得旋转体体积为:

\vspace{2ex}

    $\displaystyle{V_{ y}=2 \pi \int_{a}^{b}|x| | f(x)| d x}$

\vspace{2ex}
    
    引申1:如果绕x=c时旋转呢?

\vspace{2ex}

    $\displaystyle{V_{ y}=2 \pi \int_{a}^{b}|x-c| | f(x)| d x}$

\vspace{2ex}

    引申2:如果同时有$f_{2}(x),f_{1}(x)$绕y轴旋转的体积呢?

\vspace{2ex}

    $\displaystyle{V=2 \pi \int_{a}^{b} x f_{2}(x) d x-2 \pi \int_{a}^{b} x f_{1}(x) d x}$



\end{document}