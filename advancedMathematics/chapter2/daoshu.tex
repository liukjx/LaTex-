\documentclass[a4paper,11pt]{book}
\usepackage{xeCJK}
\usepackage{amsmath}
\usepackage[hmargin=1.25in,vmargin=1in]{geometry}
\begin{document}

\chapter*{第二章 知识框架 \quad 导数问题}  

\[
\text{求导问题}    
\begin{cases}
    \text{导数定义} \lim \limits _{x \rightarrow x_{0}}\frac{f(x)-f(x_{0})}{x-x_{0}}=\lim \limits _{x \rightarrow x_{0} }f^{\prime}(x)=f^{\prime}(x_{0})\\
    \text{求低阶导的办法}
    \begin{cases}
        \text{定义法:如某些抽象无表达式函数在某点导数,表达式分段函数在分段点处带有绝对的的函数}
        \text{求导公式法:}(\sin x)^{\prime}=\cos x\\
        \text{其他方法:如泰勒,求导困难,但展开容易}
    \end{cases}\\
    \text{求高阶导}\\
    \text{隐函数求导}
        \begin{cases}
            \text{求一次导}
                \begin{cases}
                    \text{法一:整个式子对x求导,然后处理}\\
                    \text{法二:化成}F(x,y)=0\text{型,有公式}\frac{dy}{dx}=-\frac{F^{\prime}x}{F^{\prime}y}
                \end{cases}\\
            \text{求两次导,得出一阶导的等式后对整个式子求导}
        \end{cases}\\
    \text{反函数求导}
        \begin{cases}
            \frac{dx}{dy}=\frac{1}{\frac{dy}{dx}}=\frac{1}{f^{\prime}(x)}\\
            \frac{d^{2}x}{dy^{2}}=-\frac{f^{\prime}(x)}{(f^{\prime}(x))^{2}}·\frac{1}{f^{\prime}(x)}
        \end{cases}\\
    \text{参数所决定函数求导} \frac{dy}{dx}=\frac{\frac{dy}{dt}}{\frac{dx}{dt}}=\frac{y^{\prime}_{t}}{x^{\prime}_{t}} , \frac{d^{2}y}{dx^{2}}=\frac{x^{\prime}_{t}y^{\prime\prime}_{tt}-x^{\prime\prime}_{tt}y^{\prime}_{t}}{(x^{\prime}_{t})^{3}}
    \text{参数方程是考研热点中的热点,常与以下结合考察}
        \begin{cases}
            \text{切线,法线}\\
            \text{微分方程(其中一个是)}\\
            \text{其他几何问题}
        \end{cases}\\
    \text{复合函数求导}\\
    \text{可微:主要是理解定义,有一年和泰勒一起考}\\
    \text{几何应用}
        \begin{cases}
            \text{曲率}\\
            \text{切线}
                \begin{cases}
                    1.\text{在}(x_{0},y_{0})\text{某个确切的点的切线方程}y-y_{0}=f^{\prime}(x_{0})(x-x_{0})\\
                    2.\text{在任一点}(x,y)\text{的切线方程}y-y=f^{x}(x-x)\text{变化的点}
                \end{cases}\\
            \text{法线}
        \end{cases}
\end{cases}
\]

\section*{专题一 \quad 求高阶导数问题}

法一:记住几类初等函数的高阶求导公式,要配合一些代数运算

例如:\[(\sin x)^{(n)}=\sin \left(x+\frac{n}{2} x\right)\]
$$\left(\frac{1}{a x+b}\right)^{(n)}=\frac{(-1)^{n} n!a^{n}}{(a x+b)^{n+1}}$$
$$\ln (a x+b)^{(n)}=\frac{(-1)^{n-1}(n-1) ! a^{n}}{(a x+b)^{n}}$$

\vspace{3ex}

\noindent 法二:数学归纳法(找规律)

例如:设$f(x)=e^{x} \sin x, \quad \text { 求} f^{(n)}(x)$

\vspace{3ex}

\noindent 法三:公式(适用于函数相乘,求高阶导的形式),形如求$[f(x) g(x)]^{(n)}$

\vspace{3ex}

\noindent 法四:与展开式一一对应(适用于$F^{(k)}(c)$型)c一般取0,k也可以为n
%这里缺了一个好像是泰勒展开式?

例:设$f(x)=\left(1+x^{2}\right) \ln (1+x)$,求$f^{(25)}(0)$

\section*{专题二 \quad 驻点 、极点 、最值点 、拐点}

\begin{center}
(最)极值点:x的值\\
(最)极值:y的值\\
拐点,(x,y)坐标
\end{center}


注:极值点处可导,则一定是驻点(导数为0,很多驻值是极值点,例$y=x^{3}(x=0\text{时}))$
%我有点晕,这个是什么意思,难道这个例子里这一点是极值点么,极大值还是极小值?


\subsection*{判断极值点,拐点的方法}

法一:定义法

\vspace{3ex}

\noindent 法二:
\[
\text{第一充分条件(以极值点为例)}
\begin{cases}
    \text{求出}f^{\prime}(x)\\
    \text{求出}f(x)\text{的全部驻点和不可导点(抓到两类嫌疑犯)}\\
    \text{用第一充分条件判别(挨个定罪)}
\end{cases}
\]

\vspace{3ex}

\noindent 法三:第二充分条件(适用于易求高阶导,且$x_{0}$题干告诉是哪个点,只让你判别是极小值,极大值的题目)

补充拐点的第二充分条件:若$f^{\prime\prime}(x_{0})=0,且f^{\prime\prime}(x_{0}>0(<0))$,则曲线$y=f(x)$在点$(x_{0},f(x_{0}))$的左侧邻还是凸的(凹的),右侧邻还是凹的(凸的)


\subsection*{判断是值点的方法}

%什么内,没有说清楚
最值点有可能在端点或   内取得 ,且   内的最值点也一定是极值点

步骤一:求出驻点及不可导点处的函数值\\
\noindent 步骤二:求出两个端点处的函数值   \quad    (三类嫌疑犯)\\
\noindent 步骤三:比较三类点的大小 \quad 按 大小判定
%按什么大小判定,界过

注:不需要用极值的两个充分条件去判断驻点及不可导点是否是极值,只需比较三类点的大小就好,树莓最大的就是最大值点

\section*{专题三 \quad 不等式的证明}

法一:借助求导看单调性,找最值,有时候端点处要借助求极限的方法

例如求证当$x \in (0.1)$时 $\frac{1}{\ln 2}<\frac{1}{\ln(1+x)}-\frac{1}{x}<\frac{1}{2}$


\noindent 法二:
\[
\text{构造函数求导}\footnote{观察有无拉格朗日,柯西形式决定是构造还是中值}
    \begin{cases}
        \text{情形一:形式对等时,把不同参数分列等号两侧,构造函数,}\text{例如函数求证}\\
        \qquad \qquad  e<a<b \text{时,} a^{b} \text{和} b^{a} \text{的大小关系} \\
        \text{情形二:整个式子构造函数,有些时候会换出新的元}
    \end{cases}   
\]




\noindent 法三:中值定理(拉格朗日和柯西)


\noindent 法四:凹凸性 用于两端点处等于0,二阶号恒正(负)的情况,可作图分析

\noindent 法五:二阶保号性:


如果$f^{\prime}(x)\ge 0$ 则$f(x) \ge f(x_{0})+f^{\prime}(x_{0})(x-x_{0})$

如果$f^{\prime}(x)\le 0$ 则$f(x) \le f(x_{0})+f^{\prime}(x_{0})(x-x_{0})$

\section*{专题四 \quad 求含参数的两条曲线公共点的个数(等价于求含参数方程的根的个数类问题)}
    \subsection*{题型一:只含一个未知参数a}
    方法:

    步骤一:将函数和未知数分置等号两侧,即形如$a=f(x)$(注意分母要不为零)

    步骤二:讨论含未知数一侧函数的增减情况,画出函数简图

    步骤三:让y=a与函数图相交,看交点个数不同时a的取值

    例一:试确定$e^{x}=ax^{2}(a>0)$的根的个数,并指出每个根所在的范围

    例二:求$y=k \arctan x$和y=x交点的个数

    \subsection*{题型二:含两个参数}
    
    方法:最好把两个参数都分在同一侧,另一侧是只含x的函数,如果不能就只分一个参数过去(未考过)


\section*{专题五 \quad 洛必达法则的说明}


\begin{enumerate}
    \item 必须符合$\frac{0}{0}, \frac{00}{\infty}$型才能用
    \item $f(x) \quad F(x)$必须在去心邻域内可导
    \item $\lim \limits_{x \rightarrow \alpha} \frac{f^{\prime}(x)}{F^{\prime}(x)}=A$(或无穷)时,洛必达有效,此时$\lim \limits_{x \rightarrow \alpha} \frac{f^{\prime}(x)}{F^{\prime}(x)}=A$(或无穷)
    \item $\lim \limits_{x \rightarrow a} \frac{f^{\prime}(x)}{F^{\prime}(x)}$不存在也不为无穷时,洛必达失效,此时$\lim \limits_{x \rightarrow a} \frac{f^{\prime}(x)}{F^{\prime}(x)}$存在与否不确定,需换其他方法求
    \item $\lim \limits_{x \rightarrow \alpha} \frac{f(x)}{F(x)}=A$(或无穷),不能说明$\lim \limits _{x \rightarrow \alpha} \frac{f^{\prime}(x)}{F^{\prime}(x)}=A$(或无穷)
\end{enumerate}

\section*{专题六 \quad 只含$\xi$ 的罗尔定理类问题}

\subsection*{还原法(含$\displaystyle{\frac{f^{(n-1)}(x)}{f^{(n)}(x)}}$)}

步骤一:将所证中的$\xi$ 全换成x

\vspace{2ex}

\noindent 步骤二:利用$\displaystyle{\frac{f^{\prime}(x)}{f(x)}=\left[\ln f(x)\right]} \quad (\text{或}\frac{f^{\prime\prime}(x)}{f^{\prime}(x)}=\left[\ln f^{\prime}(x)^{\prime}\right]) (\text{或}\frac{f(x)}{F(x)}=\left[\ln F(x)\right]^{\prime})$
将函数进行还原

\vspace{2ex}

\noindent 步骤三:将函数合并成$\displaystyle{\left[m \varphi (x)\right]^{\prime}=0}$,通过题干条件,$\varphi (x)$有两个函数值正好相等,$\varphi ^{\prime}(\xi )=0$经过变形正好就是要证的结论

\subsection*{分组构造法}

步骤一:雷同

\vspace{2ex}

\noindent 步骤二:可能出现 $\displaystyle{\frac{f^{\prime}(x)-k}{f(x)-kx}=\left[\ln (f(x)-kx)\right]^{\prime}},(\text{或}\frac{f^{\prime}(x)}{f(x)-x}=\left[\ln (f(x)-\lambda) \right]^{\prime})$的形式,将函数还原

\vspace{2ex}

\noindent 步骤三:雷同

\subsection*{凑微法}

步骤一:雷同

\vspace{2ex}

\noindent 步骤二:需要自己观察出所证结论的原函数,有时可借助求不定积分的办法求原函数,当然有些函数的原函数
可能是变上限函数的形式

\vspace{2ex}

\noindent 步骤三:根据题干,原函数有两个点的函数值相等,使用罗尔定理得到的中项还是要证结论

\section*{专题七 \quad 含$\xi,a,b$的中值定理 }

首先看$\xi$和a,b是否能分离,如果可以分离到等号两侧,尝试对a,b这一侧使用柯西或拉格朗日,有时候a,b可能是具体的数值

\vspace{1ex}

\noindent 如果不能分离,整个让说明的式子构造函数,去寻找其原函数,去寻找其原函数类似前一个专题讲的凑微法
%说明的式子构造函数是什么意思 

\vspace{2ex}

例一:$f(x)\in C[a,b]$在开区间(a,b)内可导,证明存在$\xi \in (a,b)$使得
$f(b)-f(a)=\xi f^{\prime}(\xi) \ln \frac{b}{a}$

例二:设a>0,b>0,(a<b)证明$\exists \xi \in (a,b)$使得$a e^{b}-b e^{a}=(a-b)(1-\xi )^{e^{\xi }}$

\section*{专题八 \quad 所证结论中含$\xi$ y(也可能同时含a,b)}

方法一

所证结论中除常数外只含$f^{\prime}(\xi ),f^{\prime}(y)$,证明方法是:找三个点,两次使用拉格朗日中值定理

\noindent 方法二

所证结论中含$\xi,y$,但设计两个中值定理的项的复杂程度不同
,将$\xi,y$分置等号两侧,让a,b分离到中值项形式简单的那一侧,等号两侧的中值项可能是经过拉格朗日或柯西得到的.(大概率中值项形式简单的那一侧使用拉格朗日定理得到的,
中值项复杂的那一侧使用柯西定理得到的不绝对)你可以用求不定积分的方法求一下每一个中值项的原函数,如果能求出原函数,那很可能就是用拉格朗日定理得到的中项

\noindent 方法三

所证结论中含$\xi, y$涉及两个中项的形式复杂程度一样(但未必完全一样)证明的方法一般是三个点分成两段,其中一段使用拉格朗日得到了第一个中项,另一段使用拉格朗日得到第二个中项,当然两个中项也有可能都是通过
柯西得到的,具体是拉格朗日还是柯西得到的中值项,还是可以用求不定积分(凑微分)的方法求一下原函数

\section*{专题九 \quad 泰勒展开的经验}

1.二阶以上的证明可以考虑泰勒展开,区间展开点不好说
,可以不停尝试,如果告诉某一点导数有等于0,可以优先考虑在该点处展开,因为展开形式变得简单

\noindent 2.如果区间对称,且告诉你两个端点函数值,则很有可能将两个端点分别在区间中点(0点)处展开,写出两个式子
%两个式子分别是啥




\end{document}